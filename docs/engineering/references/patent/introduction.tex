Presented is a set of algorithms for efficient sharing of files among devices
in a peer-to-peer configuration. These algorithms are intended to build upon the
system and methods claimed in Patent US 2012/0005159 A1~\cite{wang:patent2012},
which shall herein be referred to as the base system.

The algorithms assume such a system that identifies files and folders
(concisely referred to as {\em objects}) via globally unique object identifiers
or ids (e.g. 128-bit UUID). An {\em update} to an object signifies the
creation, deletion, modification, or renaming of a file or folder. In the base
system, a consistency algorithm exists for the propagation of these updates
among devices (i.e. eventual consistency is guaranteed). The system is modeled
to perform optimistic replication. The base system describes a {\em store} (aka
a ``Library") that is a special folder with which a specified group of users can
share and collaborate on the folder's contents. Conceptually, through object
ids, the base system maintains and synchronizes a logical representation of the
physical file system tree. Updates to the logical file system are persisted to
the physical file system.

The algorithms presented here extend the base system by enabling:
\begin{enumerate}
\item optimized network queries to download a file
\item resolution of duplication/conflict of object ids for the same file path
during a network partition
\item propagation of object deletion updates, and ability to selectively
synchronize files to a subset of devices for a given user
\item avoiding re-downloading a file when it exists locally but has remotely been
moved from one shared folder to another
\item presentation and resolution of concurrent updates to files on multiple
devices
\item one device synchronizing the stores of multiple users
\item version history for every file
\item presentation of the sync status of every file and folder
\end{enumerate}

\section*{Background}

\subsection*{Logical Object Generation}

When a device physically creates a new file or folder, the base system dictates
that the local device randomly generates a new object id to logically represent
that physical object. This local generation is applied to obviate centralized
coordination of assigning logical object ids; centralized assignment would
impede optimistic replication. Because update propagation tracks logical
objects, there must be a one-to-one mapping between a physical file/directory
and its logical object representation. The distributed approach to mapping
logical and physical objects can result in conflicts where, in a distributed
system of many devices, more than one logical object may refer to the same
physical object. This problem is addressed in the ``Aliasing" section of this
patent.
%TODO explain why logical object ids are generated locally
% * alternatives are pathnames and/or content hashing
% * these both rely on the file's own info (its name or content) but if that
%   file information changes, then the identity changes

\subsection*{Update Propagation}

The base system achieves update propagation through push and pull of {\em
version vectors}~\cite{wang:patent2012, parker:dmi1983}. A version vector is a
data structure used in optimistic replication systems; it consists of a mapping
of device ids to integer version counts per id (see the base system for further
details). For each new update, a device will push the version vector
corresponding to that change via a multicast, to a number of online devices. To
ensure no updates are missed, each device periodically pulls all other devices
for their known updates (i.e. version vectors).

\subsection*{Stores: Shared Folders}

The system permits distinct users to collaborate on a shared folder and its
contents. Such a shared folder is herein referred to as a store, which defines
the set of users who may share its contents, as specified by the store's Access
Control List (ACL). An ACL is a mapping from user id to permission on the
contents of the store. A device is permitted to sync the files and folders of a
store if its owning user id in the Access Control List (ACL) of the store.
Each device has a root store with only the owning user in the ACL; this root
store thus syncs only with devices owned by the user.

An object $o$ may be moved between stores, say $S_1$, and $S_2$. In some
algorithms, it is important to distinguish $o$ under $S_1$ vs $S_2$, and we thus
annotate $(S_1, o)$ to reference object $o$ under store $S_1$.
