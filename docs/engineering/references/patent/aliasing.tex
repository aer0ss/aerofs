\section*{Aliasing: Name-Conflict Resolution Of Object IDs}

\begin{figure}[t]
\centering
\includegraphics[width=0.75\textwidth]{figs/alias_problem.pdf}
\caption{Concurrent creation of a file generates two logical object ids for the
same path, yielding a name conflict between $o_1$ and $o_2$.}
\label{fig:aliasproblem}
\end{figure}

As stated earlier, the base file syncing system generates an object identifier
for every file and folder created on a device. The system thus maintains logical
objects that internally represent the physical objects on the local file system.
At a local device, there must be a one-to-one mapping between the logical and
physical objects, however remotely received updates can point two logical
objects to the same physical path; these {\em name conflicts} result in two
specific ways:
\begin{itemize}
\item
concurrent creation of a physical object (same pathname) on at least two
peers, resulting in more than one distinct logical object (e.g.
Figure~\ref{fig:aliasproblem}); and
\item
when one peer locally renames an existing physical object to the same path
as a physical object created on another peer.
\end{itemize}
Henceforth in this section, the term {\em object} will refer to logical object
for brevity, and physical object will be stated outright.

When peers exchange information about distinct logical objects that represent
the same-named physical object, as in Figure~\ref{fig:aliasproblem}, this
name conflict must be resolved. In the figure, devices $d_1$ and $d_2$ are
initially separated by a network partition. In that partition, they both
create a physical file (or directory) with equivalent path ``f", which yields
two distinct object ids, $o_1$ and $o_2$. Subsequently, the network partition
disappears, and device $d_1$ sends its update that object $o_1$ has been created
with path ``f". Device $d_2$ discovers that it already has a logical object
$o_2$ for path ``f", violating the invariant that logical and physical objects
must have a one-to-one mapping.

%\subsection*{Alternate resolution strategies}

One approach to resolve a name conflict is to deterministically rename one of
the two objects involved in the name conflict. However, it is conceivable
that some users of the system could already have files replicated on multiple
devices, leading to name conflicts on several files, and this rename
approach will then create $n$ copies of the same file, where $n$ is
number of peers with the replicated file. This leads to poor user experience.
Striving for a better user experience, a method is presented below to avoid
renaming and duplicating files/directories, instead opting to {\em alias} one of
the name-conflicting objects as the other.

%Another approach would be to avoid name conflicts altogether by generating the
%object id of a newly-created physical object, not randomly, but as a function
%of the path (e.g. hashing). In a scenario such as Figure~\ref{fig:aliasproblem},
%the two logical objects would be equal, and a name-conflict wouldn't exist.
%Unfortunately this generation of

\subsection*{Merging alias and target objects}

During a name conflict, specifically it is the differing object ids for the same
physical object that conflict. The claimed resolution strategy labels one of the
object ids as the {\em target}, and the other as the {\em alias}. The device
observing the name conflict merges all meta-data describing the alias object
into the target, (i.e.~consistency algorithm versions), then subsequently
replaces all meta-data about the alias object with a pointer relationship, and
shares that pointer relationship with other devices. The assignment of alias and
target objects must be deterministic, so that all devices which encounter the
same name conflict will label the same object as the target, and other objects
as aliased. To this end, since object ids can form a total order by value, the
value of object ids is used to determine the target and alias assignment.
Specifically, given a set of $n$ object ids involved in a name conflict, in one
embodiment of resolving the name conflict, the object id which will become the
target, $o_t$, has the maximum value in the set: $o_t = \max\left(\{o_1, \ldots,
o_n\}\right)$.
% TODO technically max should define over what values

\subsubsection*{Alias Version Updates}

Nearly all meta-data for the aliased object is merged into the target, including
version vectors, except for the versions associated with the aliasing operation
itself. When $o_a$ is aliased to $o_t$ at the local device, this change in state
must be propagated to all other devices to achieve a consistent global state.
Therefore a new version must be created for $o_a$, but all other versions
describing the previous file updates on $o_a$ must be merged into $o_t$, so that
$o_t$ covers the version history of its alias objects. In this patent, the
system updates version vectors by drawing from two spaces, one for alias
updates, and the other for all other object updates. When merging versions from
the alias object into the target, only versions from the non-alias space are
merged; the alias object keeps its alias version history.

In one embodiment, version vectors for the alias update space have odd-valued integer
counts, whereas version vectors for the non-alias update space have even-valued
integer counts. As with classic version vectors, the integer counts must be
monotonically increasing for events with a happens-before relationship.

\input{alias_states}
\input{alias_algo}

%\subsection*{Implementation Details}
%
%\subsubsection*{Alias vs Non-alias ticks}
%If the consistency algorithm receives a non-alias tick for a locally aliased
%object, that non-alias tick is moved to the local target object.
%\subsubsection*{Alias interaction with AntiEntropy}
%\subsubsection*{Replace n(o) with name(o), parent(o)}
%Would require describing the local "dereferencing" of a remote parent object, if
%it is aliased locally.

%%%%%%%%%%%%%%%%%%%%%%%%%%%%%%%%%%%%%%%%%%%%%%%%%%%%%%%%%%%%%%%%%
%TODO Bankim's paragraph below explains the combination of content conflict
%resolution w aliasing.  The system uses a combination of version vectors and
%content hashing to detect and resolve concurrent updates on a file viz. update
%conflict. The aliasing algorithm, along with update conflict resolution helps
%resolve name conflicts without renaming objects.
