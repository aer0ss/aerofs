\subsection*{Object State Model}

The method is first presented as a state transition model, before pseudo code
implementation. The model summarizes the initial state and expected result of the
algorithm. Defined first is a simplified model of the logical states which
can represent a physical object at a {\bf single} device. The model considers
only the metadata of a physical object, not content.
% TODO add a subsection on Content?
We begin with simple annotations, then definitions of state and transition. We
explain how they are represented in the state transition diagrams to follow.

Let $(n,o)$ represent physical object with path $n$, and logical object (id)
$o$; the object $o$ is called a non-aliased object. If logical object $o_a$ is
known to alias to $o$, then we write $o_a \rightarrow o$, knowing that $o_a$ is
an alias object for target $o$. Notice there is no path associated with $o_a$;
only the alias relationship (i.e.~\ the pointer) is stored for an aliased
object.

\subsubsection*{State}
In the case above, where object $o_a$ aliases to $(n,o)$, we write that the
state of physical object $n$ is $\{(n,o), o_a \rightarrow o\}$, which means that
file/directory $n$ is logically represented by object $o$, and any system
references to $o_a$ will alias (or are derefenced) to object $o$. Resulting from
transitions, the name-conflict resolution strategy permits only ``acceptable"
states which abide by the following two invariants:
\begin{enumerate}
\item all states must include one and only one non-aliased object; and
\item the target of all aliases must be the non-aliased object---alias ``chains"
are not permitted
\end{enumerate}
These invariants simplify the verification of correctness. The first invariant
means that a device cannot download information about an aliased object $o_a$ if
its target $o_t$ does not exist locally. This is somewhat analogous to avoiding
dangling pointers in C. The second invariant avoids creating chains of aliases,
e.g., $\{o_1 \rightarrow o_2, o_2 \rightarrow o_3\}$. Since $o_2$ is not a
non-aliased or target object locally, $o_1$ should not refer to it. In the
algorithm of the later section, when referring to a target object, one can be
sure it exists locally as a non-aliased object. In subsequent state transition
diagrams, acceptable states are represented as large circles or nodes, including
the non-aliased object, $(n,o)$, and any objects aliased to $o$. For example, at
the center of Figure~\ref{fig:aliasstate2} is a node representing the state
$\{(n,o_2), o_1 \rightarrow o_2\}$. The other nodes in the figure represent
states $\{(n,o_1)\}$ and $\{(n,o_2)\}$.

\subsubsection*{Transitions and Messages}
Since name conflicts are discovered when downloading an update about an object,
the transitions among states are spurred by messages between devices. In the
claimed name-conflict resolution method, the local device can receive two types
of messages about any object in the system from any other device:
\begin{inparaenum}[(i)]
\item a non-alias message, or
\item an alias message.
\end{inparaenum}
A non-alias message is labeled and contains meta-data of the form $(n,o)$,
implying that the sender device has provided the local device with an update
about file/directory $n$ with logical object $o$ that is not aliased on the
sender. An alias message is of the form $o_a \rightarrow o_t$, implying
that there is an update about $o_a$, and the remote device thinks its target is
$o_t$.
%TODO talk about versions?

In the subsequent state transition diagrams, an arrow (directed edge) shows the
expected transition from a state when a particular message about an object has
been received. The state transitions are agnostic about the source device.
Recall that the arrows represent transitions, not messages, but they are labeled
by the message that induces the transition. Sample transitions can be seen in
Figure~\ref{fig:aliasstate2}, such as when a device in state $\{(n,o_1)\}$
receives an alias message $o_1 \rightarrow o_2$, then subsequently transitions
to state $\{(n,o_2), o_1 \rightarrow o_2\}$.

\commented{
It is worth noting that we discovered only 13 unique types of transitions that
the algorithm must handle, and these are labeled “TA” to “TN” in Figure 2.  All
transitions fit into one of these 13 categories (a formal theorem and proof
could be made eventually), so only 13 test cases need be developed when testing
against this simplified model.
}

\subsubsection*{Object State Transition Diagrams}

The following three diagrams show all possible states that a physical object
$n$ can occupy given a replication factor, and all expected transitions across
those states.

\begin{figure}[t]
\centering
\includegraphics[width=0.75\textwidth]{figs/alias_state_2.pdf}
\caption{State transition diagram for resolving name conflicts with two distinct
object ids}
\label{fig:aliasstate2}
\end{figure}

\begin{figure}[t]
\centering
\includegraphics[width=0.6\textwidth]{figs/alias_state_3.pdf}
\caption{State transition diagram for resolving name conflicts with three distinct
object ids}
\label{fig:aliasstate3}
\end{figure}

\begin{figure}[t]
\centering
\includegraphics[width=0.75\textwidth]{figs/alias_state_4.pdf}
\caption{State transition diagram for resolving name conflicts with four distinct
object ids}
\label{fig:aliasstate4}
\end{figure}




Figure~\ref{fig:aliasstate2} considers a system where physical object $n$ has
two replicated logical objects, which are ordered such that $o_1 < o_2$. Thus
$o_2$ is the eventual target, and $o_1$ should become the alias. The center node
represents state $\{(n,o_2), o_1\rightarrow o_2\}$, which implies that the name
conflict on $(n,o_1)$ and $(n,o_2)$ has been fully resolved. The resulting
transition is shown for all three possible types of messages ($(n,o_1)$,
$(n,o_2)$, $o_1\rightarrow o_2$) from each state. Notice that if all three
messages are eventually received, the final resulting state is the
fully-resolved state.


Figure~\ref{fig:aliasstate3} considers a scenario where physical
object $n$ has three replicated logical objects, which are ordered such that
$o_1 < o_2 < o_3$. In this scenario, $o_3$ is the eventual target, with $o_1$
and $o_2$ becoming the aliases. The center node represents fully-resolved state
$\{(n,o_3), o_1\rightarrow o_3, o_2 \rightarrow o_3\}$. From this center state,
receipt of messages $(n,o_3)$, $o_1\rightarrow o_3$, $o_1\rightarrow o_2$, or
$o_2 \rightarrow o_3$ will transition back to the same state (omitted from the
diagram for brevity). The three states from Figure~\ref{fig:aliasstate2} can be
seen in the left part of Figure~\ref{fig:aliasstate3}. The state transitions
among them that were previously shown are hidden for brevity, but would
otherwise exist in this figure for completeness. The bottom three states,
$\{(n,o_2)\}$, $\{(n,o_3)\}$, and $\{(n,o_3), o_2\rightarrow o_3\}$, model
another 2-object conflict resolution, and thus share identical transitions to
those in Figure~\ref{fig:aliasstate2}, by simply replacing references to $o_2$
with $o_3$, and $o_1$ with $o_2$. Likewise with the three states in the right
side of Figure~\ref{fig:aliasstate3}. Thus in the latter figure, we opt to omit
all transitions that are described in a 2-object conflict scenario to avoid
redundancy, but they remain intrinsic to the state model.

Figure~\ref{fig:aliasstate4} extends the latter two scenarios to four replicated
objects with ordering $o_1 < o_2 < o_3 < o_4$. In this diagram, there are four
outer states with one non-alias object and two aliases, three outer states with
one non-alias and one alias, and at the center is the fully-resolved state with
one non-alias $(n,o_4)$ and three aliases to $o_4$.  As seen in the previous
name-conflict resolution state diagrams, some redundant transitions and states
are omitted in Figure~\ref{fig:aliasstate4} for clarity. The top state
$\{(n,o_3), o_1\rightarrow o_3, o_2 \rightarrow o_3\}$ is equivalent to that in
Figure~\ref{fig:aliasstate3}. All states and transitions that describe a 2- or
3-object resolution should additionally appear in this figure, but are omitted
as they are redundant when considering the previous figures. Note that the three
other 2-alias states of Figure~\ref{fig:aliasstate4} can be surrounded with
identical state hexagons with one of the objects replaced. The figure includes
states with one alias object (e.g.~\ $\{(n,o_2), o_1\rightarrow o_2\}$) because
after receipt of a particular alias message ($o_3 \rightarrow o_4$), the local
device should transition to the center, fully-resolved state. In contrast, from
a state with no alias objects (e.g.~\ $\{(n,o_2)\}$ which is not pictured),
there is no single alias nor non-alias message that will transition to the
fully-resolved 4-object state. From the center state, receipt of the following
messages will restore to the same state: $(n,o_4), o_1\rightarrow o_2, o_1
\rightarrow o_3, o_1 \rightarrow o_4, o_2\rightarrow o_3, o_2\rightarrow o_4,
o_3\rightarrow o_4$.

The presented state transition diagrams can be further extrapolated for five,
six, or more replicated objects. A scenario with five replicated objects would
have a center state with four aliases and one non-alias. The center of
Figure~\ref{fig:aliasstate4} would be among the five surrounding states with
three alias objects. One must also consider the states which can transition to
the fully-resolved state through one alias message.
