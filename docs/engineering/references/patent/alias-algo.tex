\subsection*{Aliasing Algorithm Pseudo-Code}

In one embodiment of the invention, the following algorithm facilitates the
transitions between acceptable states, providing two main routines for a device
to locally handle receipt of the two message types. Recall that the two types
of incoming messages are:
\begin{enumerate}
\item $(o, v(o), n(o), \ldots)$ an object, its version, its path, and other
  meta-data
\item $(o_t, v(o_t), n(o_t), o_a, v(o_a), \ldots)$
  an alias message with same content as above, additionally stating that
  $o_a$ aliases to $o_t$.
\end{enumerate}
Some noteworthy annotations for the following pseudocode:
\begin{itemize}
\item {\bf $vr(o) =$} version of object $o$, received remotely
\item {\bf $vl(o) =$} version of object $o$, retrieved locally
\item {\bf $nr(o) =$} name of object $o$, received remotely
\item {\bf $nl(o) =$} name of object $o$, retrieved locally
\end{itemize}
Logical objects must represent the same type of physical object in order to be
aliased (i.e.~\ both the remote and local objects must be a file or both must be
a directory).

\subsubsection*{Common Routines}

Both message handlers share a common routine to logically receive a non-alias
object that was previously locally unknown. If receipt of the new object will
create an alias object, the argument $o_{nonewvers}$ is used to decide whether
to publish a new version/update about the alias action. Some callers of this
handler do not need to publish the creation of the alias object, such as when
an alias message is being processed; the update for the aliasing operation was
created at some other device, hence why the local device is now processing an
alias message.
\begin{algorithmic}
\Function{ReceiveNewObject}{$o$: object id, $vr(o)$, $nr(o)$, $o_{nonewvers}$}
\If{locally $\exists o_{nr}$
such that $nr(o) == nl(o_{nr}) $ and $o \neq o_{nr}$}
    \Comment{a name conflict exists}
    \If{have not queried sender about $o_{nr}$}
        \State query sender for meta-data of $o_{nr}$ and retry $o$ later
        \State \Return
    \EndIf
    \If{$o_{nr} \neq \max(o, o_{nr})$}
        \State swap($o$,$o_{nr}$)
        \Comment{ensure $o_{nr}$ is the target}
    \EndIf
    \State MergeAliasIntoTarget($o$, $o_{nr}$)
    \If{$o \neq o_{nonewvers}$}
        %\State \Comment{Create a new version for $o$ to notify other devices
            %that $o$ has been aliased}
        \State $vl(o) =$ new version from alias version space
    \Else
        \State $vl(o) = \emptyset$
        \Comment{the caller has indicated that the version of $o$ need not be
            incremented}
    \EndIf
\Else
    \State perform typical non-name-conflict execution
\EndIf
\EndFunction
\end{algorithmic}
The latter routine defers to the following subroutine to merge the alias and
target objects.
\begin{algorithmic}
\Function{MergeAliasIntoTarget}{$o_a$, $o_t$}
\State merge non-alias versions of $o_a$ into $o_t$
\If{$o_a$ is a file}
    \State move conflicting content branches from $o_a$ to $o_t$
    \State merge content branches with identical content in $o_t$
\ElsIf{$o_a$ is a directory}
    \State move children of $o_a$ into $o_t$, renaming if a name-conflict would
result
    \State delete physical directory of $o_a$
\EndIf
\State locally record $o_a \rightarrow o_t$
\For{$o \in \{o | o\rightarrow o_a$ is recorded locally$\}$}
    \State replace $o\rightarrow o_a$ with $o\rightarrow o_t$
    \Comment{must resolve chaining of existing alias objects}
\EndFor
%\State delete remaining meta-data for $o_a$
\EndFunction
\end{algorithmic}

\subsubsection*{Non-Alias Messages}

The non-alias message handler is relatively straightforward, in some branches
dispatching to the standard file syncing algorithm, not the name-conflict
resolution strategy.
\begin{algorithmic}
\Function{HandleNonAliasMsg}{$o$, $vr(o)$, $nr(o)$}
\If{locally $o$ is non-aliased}
    \State perform typical non-name-conflict handling
\ElsIf{locally $\exists o_t$ such that $o\rightarrow o_t$}
    \State no-op
    \Comment{remote updates on $o$ will later be sent via $o_t$}
\Else
    \State ReceiveNewObject($o$, $vr(o)$, $nr(o)$, null)
\EndIf
%\State respond success to sender for message processing
\EndFunction
\end{algorithmic}

\subsubsection*{Alias Messages}

The alias message handler is more involved. As stated at the beginning of this
section, the data of a non-alias message is actually embedded in an alias
message, where this data represents the target. This protocol is required to
satisfy the first invariant for our object states: a consistent state for
physical object $n$ must logically have one non-aliased object. When an alias
message is received such that the target was previously unknown to the local
device, the target object must be accepted into the local system before
processing the aliased object. This represents the first phase of the routine.
With the target object handled correctly, the alias object is then processed.

\begin{algorithmic}
\Function{HandleAliasMsg}{$o_t$, $vr(o_t)$, $nr(o_t)$, $o_a$, $vr(o_a)$}
\If{locally $o_t$ is not known}
    \Comment{as aliased or non-aliased object}
    \State ReceiveNewObject($o_t$, $vr(o_t)$, $nr(o_t)$, $o_a$)
    \Comment{do not create new version for $o_a$}
\EndIf
\If{locally $\exists o_e$ such that $o_t\rightarrow o_e$}
    \State $o_t = o_e$
    \Comment{avoid alias chains; reset $o_t$ to its target}
\EndIf
%\Ensure{locally $o_t$ is now a non-aliased object}
\If{locally $o_a$ is non-aliased}
    \State MergeAliasIntoTarget($o_a$, $o_t$)
\ElsIf{locally $\exists o_e$ such that $o_a \rightarrow o_e$}
    \If{$o_e \neq o_t$}
        \If{$o_t \neq \max(o_e, o_t)$}
            \State swap($o_t$, $o_e$)
        \EndIf
        \State MergeAliasIntoTarget($o_e$, $o_t$)
        \Comment{resolve alias chain of $o_a$, $o_e$, and $o_t$}
    \EndIf
\Else
    \Comment{locally $o_a$ is not known}
    \State locally record $o_a\rightarrow o_t$
    \State $vl(o_a) = \emptyset$
\EndIf
\State $vl(o_a) = vl(o_a) \cup vr(o_a)$
\EndFunction
\end{algorithmic}

%\begin{itemize}
%\item comment on subtleties of branches as in the aerofs algo-doc?
%\item describe version vector union?
%\item describe persistence of objects in OA table?
%\item describe persistence of alias table?
%\end{itemize}
