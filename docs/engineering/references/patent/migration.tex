\section*{Migration: moving files across store boundaries}

Recall the concept of a store from the base system~\cite{wang:patent2012} and
Background section of this patent. A store defines a set of users who are
sharing a directory and its contents. Moving an object between stores deserves
special consideration. The system supports the ability to delete files when
moved out of a store, or move files among stores, depending on the context. The
problem is illustrated in Figure~\ref{fig:migrationgoal}. Additionally, the
system maintains cross-store version history, providing a causal history for an
object that crosses store boundaries, as seen in
Figure~\ref{fig:migrationhistory}.

\begin{figure}[t]
\centering
\includegraphics[width=0.75\textwidth]{figs/migration_goal.pdf}
\caption{Expected behavior following a migration across stores for devices
subscribing to a different set of stores}
\label{fig:migrationgoal}
\end{figure}

Figure~\ref{fig:migrationgoal} shows the state of two stores, $S_1$ and $S_2$
on four devices, $d_1, d_2, d_3, d_4$, after moving an object between stores.
Devices $d_1$ and $d_2$ are subscribed to both stores. Device $d_3$ is
subscribed to $S_1$ only, and $d_4$ to $S_2$ only. Initially object $o_1$ is
under the root directory of store $S_1$, and all devices are consistent with
this state. Device $d_1$ moves $o_1$ into store $S_2$. The system supports the
following state transitions when each of devices $d_2, d_3,$ and $d_4$ receives
the update of the cross-store object movement.
\begin{itemize}
\item on $d_2$, the object is physically moved, without deleting and
re-downloading the content of $o_1$;
\item on $d_3$ the object is physically deleted; and
\item on $d_4$ the object is downloaded and physically created.
\end{itemize}
The Collector, Expulsion, and update propagation algorithms are store-centric,
thus what should be a simple move operation between stores on $d_2$ could be
naively implemented atop these algorithms as separate deletion and creation
operation. A device receiving the deletion and creation updates would thus
naively re-download the file, even if the content had not changed. Through the
method of {\em migrating} a logical object between stores, the system avoids
naively deleting the object from $S_1$, then re-downloading the object into
$S_2$.

\begin{figure}[t]
\centering
\includegraphics[width=0.6\textwidth]{figs/migration_history_goal.pdf}
\caption{An update to the content of $o_1$ under store $S_1$ is successfully
propagated, despite $o_1$ being moved to store $S_2$ concurrently}
\label{fig:migrationhistory}
\end{figure}

Figure~\ref{fig:migrationhistory} illustrates the goal for cross-store version
history. Initially, object $o_1$ is consistent under store $S_1$, on both
devices $d_1$ and $d_2$. During a network partition, device $d_2$ modifies the
content of $o_1$ (indicated by the modified pattern of the node), but leaves the
object in store $S_1$. Concurrently, device $d_1$ moves the object to store
$S_2$ with the original content. The network partition then disappears and the
two devices propagate their updates. By maintaining the identity of $o_1$ across
stores, and maintaining the version history, a final state can be achieved where
$o_1$ is under store $S_2$ with the new content applied while it was under
$S_1$. Without tracking identity and the cross-store version history, there
would be two files, one in each store, and only one would have the updated
content.

The system observes the migration of the object id $o_1$ and maintains the
consistency algorithm version history of the file, through
\begin{inparaenum}[(i)]
\item respecting the invariant that a given object can be admitted in only one
store at any time (as in the Expulsion algorithm), and
\item an extension to the versioning system of the consistency algorithm, called
{\em immigrant versions}.
\end{inparaenum}

\subsection*{State Change When Instigating Migration}

\begin{figure}[t]
\centering
\includegraphics[width=\textwidth]{figs/migration_instigator.pdf}
\caption{}
\label{fig:migrationinstigator}
\end{figure}

Figure~\ref{fig:migrationinstigator} shows the logical state change after an
object $o_1$ is physically moved between stores $S_1$ and $S_2$ on the local
device. Initially object $o_1$ with name $n$ is under the root folder of $S_1$,
and $S_2$ has no child object. Following the physical move, under store $S_1$
the object is effectively deleted as indicated in the Expulsion section of this
patent, by logically being moved under the trash folder, and thus expelled.
Notably, under the trash folder, the name of the object is the store id of
$S_2$, the target store to which $o_1$ was moved. Under store $S_2$, object
$o_1$ is created in the admitted state. As with the usual consistency algorithm,
these two logical state changes generate two updates which are propagated to
other devices:
\begin{itemize}
\item $o_1$ was moved under the $S_1$ trash folder with name $S_2$, and
\item $o_1$ was created under the $S_2$ root folder with name $n$.
\end{itemize}
The physical object maintains its logical object id across stores in an effort
to easily identify migration, and maintain its version history despite store
migrations.

\subsection*{On Deletion Update}

Consider a device which subscribes to store $S_1$; it will receive
the first update, that $o_1$ was moved to the trash folder. Because the new name
of $o_1$ under store $S_1$ identifies the store to which $o_1$ {\em emigrated},
the device receiving this update can infer the new store of $o_1$. The following
pseudo-code enumerates the steps taken to migrate an object $o$ on receipt of a
deletion update under store $S$.
\begin{algorithmic}
\Function{HandleDeleteUpdate}{$S$: store id, $o$: object id: $n$: name}
\If{$o$ is locally admitted in $S$ and has not been emigrated and $n$ encodes a store id}
    \State $S_{target} \gets$ decoded store id from $n$
    \State attempt download of $S_{target}$ and its ancestor stores from remote
    \If{store $S_{target}$ is not present locally}
        \State \Return
            \Comment{do not migrate as the target store cannot be downloaded}
    \EndIf
    \If{$o$ is a file}
        \State download $o$ under $S_{target}$ from remote
    \Else
        \State download $o$ and its children under $S_{target}$ from remote
    \EndIf
\Else
    \State process the deletion update for $o$ as usual
\EndIf
\EndFunction
\end{algorithmic}
This method to handle migration-induced deletions determines the target
store of the object to be migrated, then defers to the handler of creation
updates (to follow). Because migrated objects keep the same logical identifier,
once the deletion handler has determined the target store, it can simply request
the object under that store. A non migration-induced deletion will be
handled by the Expulsion algorithm described earlier.

\subsection*{On Creation Update}

Now consider a device which subscribes to store $S_2$; it receives the second
update, that $o_1$ was created under $S_2$ with name $n$. In a typical work
flow the object is physically downloaded, but to avoid redundant transfers,
the local device first determines whether $o_1$ is admitted in any other
store. If $o_1$ is admitted in another store, the local device migrates
$o_1$ under $S_2$ by physical file movement. The following pseudo-code
details the steps taken to migrate an object $o$ on receipt of a creation
update under store $S$. The annotation $(S,o)$ is used to identify object $o$
under $S$. The creation of immigrant versions is explained in the following
subsection.
\begin{algorithmic}
\Function{HandleCreationUpdate}{$S$: store id, $o$: object id: $n$ name}
\If{$\exists S_{source}$ where $o$ is locally admitted in store $S_{source}$}
    \For{each content conflict branch $b$ of $(S_{source},o)$}
        \State copy modification time, file length, content hash of $b$
               to $(S,o)$
        \State physically move the branch to its corresponding location
                under $S$
    \EndFor
    \State create immigrant versions from $(S_{source},o)$ to $(S,o)$
    \State move $o$ in $S_{source}$ under the trash folder
    \State logically rename $o$ to an encoding of $S$
\Else
    \State process the creation update for $o$ as usual
\EndIf
\EndFunction
\end{algorithmic}

Note that the creation update handler concludes by deleting $o$ from the source
store, and recording its migrated target store. This action implicitly will
create a new version update for $(S_{source},O)$ on the local device, which will
be propagated to other devices. However all devices that subscribe to the target
store $S$ will perform the same action, generating false version conflicts. The
base system~\cite{wang:patent2012} discusses an approach to automatically
resolve such conflicts.

\subsection*{Immigrant Versions: Cross-Store Consistency}

The background section of this patent briefly summarizes update propagation in
the base system~\cite{wang:patent2012}. This section is mainly concerned with
pull requests. Naively, a device could respond to a pull request by sending its
entire local set of version vectors. However, the two devices may share many of
those versions, resulting in much redundant bandwidth waste. The base system
further proposes the concept of stability of version vectors, by defining a {\em
knowledge vector}, present locally on a device, one knowledge version vector for
every store. All integer counts below the knowledge vector of the local device
are assumed to be stable---no new version needs to be requested whose integer
count is below the knowledge vector. Because of this invariant, when issuing a
pull request, a device X can send its knowledge vector to device Y, and device Y
can respond with only those versions which are above the given knowledge vector.
Additionally, device Y responds with its knowledge vector after the version
exchange so that device X can increase its own vector accordingly. See the prior
patent for examples and details.

The migration of an object across stores requires special consideration with
regard to the knowledge vector. The system cannot simply move versions for an
object in one store to its new target store, as this naive approach could move
a version from the source store to the target store that is under the knowledge
vector of the target store. This in turn would break the invariant that all
versions below a knowledge vector are stable, and thus we could no longer assume
that pull-based update propagation will guarantee to receive all new updates
from the pulled device.

To ensure that pull-based update propagation guarantees the propagation of all
updates in the face of store migration, this patent presents {\em immigrant
version vectors}. Whereas each regular version vector ({\em native} version
vector) is associated with the update of one logical object, an immigrant
version vector is associated with the migration of a native version vector.
The concurrency control subsystem thus has two version management systems, one
for native versions which track object updates, and one for immigrant versions,
which track native version. Immigrant versions similarly have a knowledge
vector, and stability of immigrant versions. When an object $o$ is locally
migrated from store $S_{s}$ to $S_t$ on device $d$, a new immigrant version is
created for $o$ on $d$, recording the version of $o$ that was migrated from
$S_s$. As part of the pull-based update propagation strategy, immigrant versions
are requested that are above the immigrant knowledge vector. If a received
immigrant version was previously unknown to the local device, then the native
version tracked by the immigrant version is persisted in the local device's native
version table. The immigrant version subsystem can thus insert native versions
under the native knowledge vector, but no native versions are at risk of loss
because of cross-store object migration.

%\begin{itemize}
%\item immigrant max tick is bumped
%\item immTick = tick above
%\item immDid = local did
%\item (did, tick) = original pair to be migrated from old store
%\item associated with $(S_t, o)$.
%\end{itemize}
%How is it used...
%
%\begin{itemize}
%\item GetVersCall rpc: send kwlg and imm kwlg
%\item GetVersCall process: reply with all versions above remote kwlg and
%immigrant versions above remote imm kwlg
%\item GetVersReply process: if receive immigrant tick that is new to this local
%device, assign the tick/did pair (not the immigrating device/tick), to the
%target object. TODO determine why that is safe, and what it signifies
%\end{itemize}
